\documentclass[a4paper, 12pt]{report}

% Gestion des packages
\usepackage[utf8]{inputenc}
\usepackage[T1]{fontenc}
\usepackage[french]{babel}
\usepackage[a4paper]{geometry}
\usepackage{graphicx}
\usepackage{blindtext}
\usepackage{fancyhdr}

% Propriétés du document

\sloppy

\title{BAB 1 - Sciences Informatiques \\  Rapport du Projet d'informatique 2018 - 2019 \\ Création d'un jeu de plateau Stratego\thanks{Stratego est un jeu de société de stratégie et de bluff, créé en 1947 (dérivé du jeu L'Attaque, breveté par Hermance Edan en 1909). \underline{https://fr.wikipedia.org/wiki/Stratego}}}
\author{Oussama Ouanane}
\date{2019\\ Vendredi 17 Mai}

\begin{document}

% Titre

\begin{titlepage}
\clearpage\maketitle
\thispagestyle{empty}
\end{titlepage}

% Table des matières

\begin{table}
\tableofcontents
\end{table}
\part{Avant-jeu}
\chapter{Présentation du jeu}
Le Stratego a l'allure d'un jeu complexe mais il n'en est rien. Il se joue à deux personnes et est ce qu'on appelle dans le jargon des jeux-vidéos, un \textit{Turn-based game}. Chaque joueur possède 40 pions


\end{document}